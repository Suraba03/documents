\documentclass[a4paper,12pt]{article}
\usepackage{ upgreek }
\usepackage{ tipa }
\usepackage[T2A]{fontenc}
\usepackage[utf8]{inputenc}
\usepackage[english,russian]{babel}
\usepackage{amsmath,amsfonts,amssymb,amsthm,mathtools}
\usepackage{hyperref}
\usepackage{blindtext}


\date{}
\begin{document}


\section{О себе}
      \large Чекменев Вячеслав Алексеевич, 20 лет\\
      \large почта: chekmenev031@gmail.com \\
      \large телефон: +7-985-829-15-40 \\
      \large \href{https://github.com/Suraba03}{GitHub} \\
      \large Telegram: @suraba03\\
\section{Технологии}
\begin{enumerate}
    \item \underline{Python}: работа в jupyter notebook и google colab. \textbf{Scikit-learn, SciPy, pandas, numpy, sympy, matplotlib, django}, pyTorch, OpenCV, tenzorflow, научные вычисления, веб-приложения. \textbf{Топологический анализ данных}
    \item \textbf{Математика}: Мат. анализ, лин. алгебра, теория вероятностей и математическая статистика, методы оптимизации, вычислительная геометрия и топология
    \item \underline{С} и \underline{С++}: \textbf{структуры данных, основные алгоритмы}, истемное программирование под UNIX. ООП, функциональное программирование, \textbf{олимпиады на уровне div. 2} 
    \item C++ для анализа \textbf{временных рядов}. Например, производных \textbf{финансовых инструментов}.
    \item Логическое программирование на \underline{Prolog} для \textbf{NLP}
    \item \underline{Веб-программирование на JavaScript} - технологии: \textbf{node.js, express.js, mongodb, postgres, docker, react, vue.js}
    \item \textbf{Latex}. Формулы, форматирование документов.
    \item Работа с \textbf{git, github, gitlab}, контейнеризация с \textbf{docker}

\end{enumerate}

\section{Проекты}
\begin{itemize}
    \item \textbf{Computer vision} "Создание модели машинного обучения для улучшения качества старых фильмов (enhancement)".

    \item На этапе ресерча проект по тегированию изображений с использованием \textbf{топологического анализа данных и OpenCV}.
    
    \item Анализ \textbf{временных рядов}. Расчет различных кривых и \textbf{математических моделей финансовой математики} (26 функций) на языке С++, описание и код можно найти \href{https://github.com/Suraba03/quantaton-2022-preparation}{тут}

    \item \textbf{NLP} "Генеалогическое дерево семьи Шекспира и анализ запросов к нему с использованием языка prolog". \href{https://github.com/Suraba03/LP/tree/main/cp}{код}

    \item "Веб-приложение для сборки компьютера \textbf{стек: django, postgres, docker}". \href{https://github.com/Leha-Slava-Max-Kirill/computer_workshop}{код}
    
    \item "Машинное обучение. \textbf{Decision trees и random forests} и  для предсказания факта успешного окончания MOOC-курса обучающимся". \href{https://github.com/Suraba03/mini_ML_project_sem1}{код}
    
    \item "Алгоритм Дейкстры поиска кратчайших путей в графе. Графический пользовательский интерфейс.". \href{https://github.com/Suraba03/mai_prooga_sem2/tree/main/cp_dm_Dijkstra}{код}
    
    \item "Многопоточная клиент-серверная система для мгновенной или отложенной отправки сообщений (\textbf{мессенджер}) на \textbf{С++}". \href{https://github.com/Suraba03/OS_MAI/tree/main/cp}{код}

    \item "Создание \underline{многопоточного} веб-приложения на \textbf{JavaScript}". \href{https://github.com/Leha-Slava-Max-Kirill/application}{код}

\end{itemize}

\section{Soft skills и разные навыки}
\begin{itemize}
    \item \textbf{Английский язык B2}. Могу читать техническую литературу, смотреть курсы, общаться.
    \item \textbf{Организоваю} клуб по олимпиадной математике, решаю вопросы по клубу настольного тенниса в вузе, ищу соревнования и прочее.
    \item Организовываю однокурсников для подготовки к контрольным, коллоквиумам, экзаменам.
    \item Занимаюсь \textbf{спортом}, Умею работать в команде и \underline{брать ответственность}.
\end{itemize}

\section{Чем занимаюсь для улучшения навыков}
\begin{itemize}
    \item Участвую в \textbf{олимпиадах} по программированию на codeforces в div2. Также \textbf{rucode} и \textbf{vkcup}
    \item В олимпиадах по математике и информатике: Математическая регата Тинькофф, Я - профессионал, Высшая лига, математика машинного обучения

    \item Прохожу разные \textbf{курсы} на степике и подобных платформах.
    \\ Обучающие курсы на \underline{kaggle} и от google по \textbf{нейронным сетям}.
    \\Получил сертификаты по курсам:        \href{https://stepik.org/course/363/syllabus}{Введение в программирование (С++)}, \href{https://stepik.org/course/4852/syllabus}{Введение в \textbf{Data Science} и машинное обучение},          \href{https://stepik.org/course/3078/syllabus}{Основы программирования на C. Задачи.}, \href{https://stepik.org/course/76/syllabus}{Основы статистики}, \href{https://stepik.org/course/524/syllabus}{Основы статистики. Часть 2}    
\end{itemize}

\end{document}
