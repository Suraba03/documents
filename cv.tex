\documentclass[a4paper,12pt]{article}
\usepackage{ upgreek }
\usepackage{ tipa }
\usepackage[T2A]{fontenc}
\usepackage[utf8]{inputenc}
\usepackage[english,russian]{babel}
\usepackage{amsmath,amsfonts,amssymb,amsthm,mathtools}
\usepackage{hyperref}
\usepackage{blindtext}


\date{}
\begin{document}

\begin{titlepage}
   \vspace*{\stretch{1.0}}
   \begin{center}
      \Large\textbf{Резюме.}\\
      \large\textit{ Чекменев Вячеслав Алексеевич, 19 лет}\\
      \large почта: chekmenev031@gmail.com \\
      \large телефон: +7-985-829-15-40 \\
      \large \href{https://github.com/Suraba03}{GitHub} \\
      \large Telegram: @suraba03\\
   \end{center}
   \vspace*{\stretch{2.0}}
\end{titlepage}

\section{Образование}
\begin{itemize}
    \item Сейчас получаю высшее образование (бакалавриат):
        \begin{enumerate}
            \item В "\textbf{НИУ МАИ}"
            \item По программе \textbf{Прикладная Математика и Информатика} (01.03.02)
            \item Курс 3
            \item Начал обучение в \underline{2020} году, закончу в \underline{2024}.
        \end{enumerate} 
    \item Также я закончил \textbf{"Бауманскую инженерную школу №1580"} по \underline{физико-математическоаму профилю} (инженерный класс в московской школе)
\end{itemize}

\section{Какими технологиями я владею}
\begin{enumerate}
    \item Программирование на \underline{С} и \underline{С++}: \textbf{структуры данных, основные алгоритмы} на строки, графы, деревья, сортировки, также жадные алгоритмы, динамическое программирование; системное программирование под UNIX. ООП, немного функционального программирования, также пишу \textbf{олимпиады на уровне div. 2} 
    \item Программирование на \underline{python}: работа в jupyter notebook и google colab. \textbf{Библиотеки OpenCV, tenzorflow, Scikit-learn, SciPy, pyTorch, pandas, numpy, sympy, matplotlib, django}, научные вычисления, веб-приложения
    \item C++ для анализа \textbf{временных рядов}. Например, производных \textbf{финансовых инструментов}. Hull–White model, Vasicek model, Black–Karasinski model, yield curves.
    \item Логическое программирование на \underline{Prolog} для простенького \textbf{NLP}
    \item \underline{Веб-программирование на JavaScript} - технологии: \textbf{node.js, express.js, mongodb, postgres, docker, react, vue.js}
    \item Программирование на \underline{C\#}: \textbf{формальные языки, грамматики}, конструирование компилятора и транслятора
    \item \textbf{Топологический анализ данных} на python, библиотека \textbf{GUDHI}
    \item \textbf{Latex}. Формулы, форматирование документов.
    \item Работа с \textbf{git, github, gitlab}
    \item Контейнеризация с \textbf{docker}
\end{enumerate}

\section{Личные проекты}
\begin{itemize}
    \item Анализ \textbf{временных рядов}. Расчет различных кривых и \textbf{математических моделей финансовой математики} (26 функций) на языке С++, описание и код можно найти \href{https://github.com/Suraba03/quantaton-2022-preparation}{тут}
    \item На этапе ресерча проект по тегированию изображений с использованием \textbf{топологического анализа данных и OpenCV}.
\end{itemize}

\section{Математика}

Математику знаю на уровне курсов технических университетов в рамках прикладной математики. 
 Изучал по: \\
\\

\begin{tabular}{ | l | l | l | }
    \hline
        предмет & по чему изучал теорию & из чего прорешал задачи \\ \hline
        Мат. анализ & Зорич; Кудрявцев & Демидович \\
        алгебра & Городенцев, Кострикин & Городенцев,  Кострикин \\
        лин. алгебра & Кострикин & Кострикин \\
        Дифф. уравнения & Эльсгольц & Филиппов \\
        Теория вероятностей & Гмурман & Гмурман \\
        Топология & Виро & Виро\\
    \hline
\end{tabular}
\subsection{Сейчас изучаю}
\begin{itemize}
    \item \textbf{Алгебраическая топология}
    \item Алгоритмы топологического анализа данных
    \item Базовые области математики на более глубоком уровне: действительный анализ, алгебра, геометрия.
\end{itemize}

\section{Мои Soft skills и разные навыки}
\begin{itemize}
    \item \textbf{Организоваю} неформальный клуб по олимпиадной математике, собираемся раз в неделю.
    \item Решаю вопросы по клубу настольного тенниса в вузе. Доставляю информацию, ищу соревнования и прочее.
    \item Организовываю однокурсников для бота контрольных, коллоквиумов, экзаменов.
    \item Много занимаюсь \textbf{спортом}, нравятся такие виды как: волейбол, настольный теннис, бильярд.
    \item \underline{Умею работать в команде и брать на себя ответственность}.
\end{itemize}

\section{Чем занимаюсь помимо ВУЗа}
\begin{itemize}
    \item С первого курса ВУЗа участвую в \textbf{олимпиадах} по программированию на codeforces в div2. Также \textbf{rucode} и \textbf{vkcup}
    \item В этом году участвую в таких олимпиадах как:
        \begin{enumerate}
            \item \textbf{Математическая регата Тинькофф}
            \item \textbf{Я - профессионал}: математика, программная инженерия, информационные технологии, искусственный интеллект
            \item \textbf{Высшая лига}: математика, прикладная математика и информатика.
            \item \textbf{Поволжская математическая олимпиада студентов}.
            \item Также буду в этом году писать олимпиаду \textbf{"математика машинного обучения"}
        \end{enumerate}
    \item Прохожу разные \textbf{курсы} на степике и подобных платформах.
    \\ Обучающие курсы на kaggle и от google по \textbf{нейронным сетям}.
    \\Получил сертификаты по курсам:
        \begin{enumerate}
            \item \href{https://stepik.org/course/363/syllabus}{Введение в программирование (С++)}
            \item  \href{https://stepik.org/course/4852/syllabus}{Введение в \textbf{Data Science} и машинное обучение} 
            \item \href{https://stepik.org/course/3078/syllabus}{Основы программирования на C. Задачи.}
        \end{enumerate}
    Без сертификатов:
        \begin{enumerate}
            \item \href{https://stepik.org/course/76/syllabus}{Основы статистики}
            \item \href{https://stepik.org/course/524/syllabus}{Основы статистики. Часть 2}
        \end{enumerate}
    \item Посещаю занятия НМУ по алгебре и анализу, решаю листки
    
\end{itemize}



\section{Опыт научной и практической работы в ВУЗе}

\subsection{Курсовые проекты в ВУЗе}
\begin{enumerate}
    \item \textbf{NLP} "Генеалогическое дерево семьи Шекспира и анализ запросов к нему с использованием языка prolog". \href{https://github.com/Suraba03/LP/tree/main/cp}{код}
    
    \item "Веб-приложение для сборки компьютера \textbf{стек: django, postgres, docker}". \href{https://github.com/Leha-Slava-Max-Kirill/computer_workshop}{код}
    
    \item "Машинное обучение. \textbf{Decision trees и random forests} и  для предсказания факта успешного окончания MOOC-курса обучающимся". \href{https://github.com/Suraba03/mini_ML_project_sem1}{код}
    
    \item "Алгоритм Дейкстры поиска кратчайших путей в графе. Графический пользовательский интерфейс.". \href{https://github.com/Suraba03/mai_prooga_sem2/tree/main/cp_dm_Dijkstra}{код}
    
    \item "Многопоточная клиент-серверная система для мгновенной или отложенной отправки сообщений (\textbf{мессенджер}) на \textbf{С++}". \href{https://github.com/Suraba03/OS_MAI/tree/main/cp}{код}

\end{enumerate}
\subsection{Летние практики}

\begin{enumerate}
    \item "Создание \underline{многопоточного} веб-приложения на \textbf{JavaScript}". \href{https://github.com/Leha-Slava-Max-Kirill/application}{код}
    \item "Создание модели машинного обучения для улучшения качества старых фильмов (enhancement)".
\end{enumerate}

\end{document}
